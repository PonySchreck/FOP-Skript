\chapter{Rekursive Datentypen und Strukturelle Rekursion}
\section{Listen}
Mit strukturen können Datenobjekte mit einer festen Zahl von Daten gespeichert werden. Häufig wissen wir jedoch nicht, aus wie vielen Datenelementen eine Datenstruktur besteht.
%%%%%%%%%%%%%%%%%%%%%%%%%%%%%%%%%%%%%%%%%%%
Oder die struktur der Daten ist rekursiv
%???? KEINE AHNUNG WAS MIT DEM LETZTEN SATZ GEMEINT IST!!!
Mit rekursiven Datentypen können auch beliebig große Datenobjekte strukturiert abgespeichert werden. Die Idee davon ist die Folgende: Ein Element der Datenstruktur speichert (direkt oder indirekt) ein Exemplar der Datenstruktur. Dies nennt man dann eine \textit{rekursive Datenstruktur}. Um eine \uline{endliche} Datenstruktur zu bekommen benötigt man einen \textit{Rekursionsanker}. Diesen Rekursionsanker modellieren wir mit der Technik zu heterogenen Daten aus dem letzten Kapitel.

\section{Modellierung eines rekursiven Datentyps}

Eine Liste ist entweder die leere Liste the-emtpylst, oder (amke-lst s r), wobei s ein Wert ist und r eine Liste.

\begin{lstlisting}{t03-prog1}
;; a list with 0 elements
;; (define list0 the-emptylst)
(define list0 empty)

;; a list with 1 element
;; (define list1 (make-lst 'a the-emptylst))
(define list1 (cons 'a empty))

;; a list with 2 elements
;; (define list2 (make-lst 'a
;;               (make-lst 'b the-emptylst)))
(define list2 (cons 'a (cons 'b empty)))

;; get the 2nd element from list2
;; (lst-first (lst-rest list2)) -> 'b
(first (rest list2)) ;; -> 'b
\end{lstlisting}
