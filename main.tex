


% Dokumentenklasse
\documentclass[a4paper,12pt,liststotoc, parskip=half]{scrreprt}

    % ============ Pakete ============
    % Dokumentinformationen

    % FONTS
    \usepackage[utf8]{inputenc}				% UTF-8 Zeichensatz
    \usepackage[ngerman]{babel}				% Alle Bezeichnungen auf die deutsche Sprache anpassen
    \usepackage[T1]{fontenc}				% Unterstützung für westeuropäische Codierung(Umlaute)


    \usepackage{fancyhdr}					% Einfache Bearbeitung von Kopf- und Fußzeile

    \usepackage{graphicx}					% Grafiken einbinden
    \usepackage{subfig}						% Abbildungen und Tabellen
    %\graphicspath{{Anhang/bilder/}}		% Pfad zu Grafiken

    \usepackage{lmodern}					% Verändert die Schriftart auf "Latin Modern"
    \usepackage{color}						% Farbenmanagement
    \usepackage{booktabs}

    \usepackage[onehalfspacing]{setspace}	% Abstände
    \usepackage{geometry}					% Seiten-Layout
    \usepackage{authblk}
    \usepackage{longtable}
    % Zusätzliche Schriftzeichen der American Mathematical Society
    \usepackage{amsfonts}
    \usepackage{amsmath}

    \usepackage{pdfpages}

    %Erweiterte Referenzen
    \usepackage{hyperref}
    \usepackage[ngerman, nameinlink, noabbrev]{cleveref}

    % C++ Code Gedöns
    \usepackage{listings}
    \lstset{numbers=left, numberstyle=\tiny, numbersep=5pt}
    \lstset{language=C}

    % Fortlaufende nummerierung
    \usepackage{chngcntr}
    \counterwithout{figure}{chapter}
    \counterwithout{table}{chapter}

    % Underlining
    \usepackage{ulem}

    \usepackage{fancybox}
    % Style
    \pagestyle{fancy}

    % Kopfzeile
    \lhead{ }
    \chead{ }
    \rhead{ }

    % Fußzeile
    \lfoot{ \leftmark}
    \cfoot{\slshape }
    \rfoot{\thepage}

    % Trennlinien
    %\renewcommand{\headrulewidth}{0.0pt}	% keine trennlinie nach der Kopfzeile
    \renewcommand{\footrulewidth}{0.2pt}		% Dünne Trennline vor der Fußzeile

    % ============ Paketeinstellungen & Sonstiges ============
    % Besondere Trennungen
    \hyphenation{De-zi-mal-tren-nung}
    \hyphenation{Ent-wick-lung}
    \hyphenation{ent-wick-eln}
    \hyphenation{Library}

    %Helvetia FONT
    \usepackage{mathptmx}
    \usepackage{helvet}

    % ============ Dokumentbeginn ============

    %\setmainfont{Calibri}
    \begin{document}
    % Seiten ohne Kopf- und Fußzeilen sowie Seitenzahl
    \pagestyle{fancy}
    %\include{titel}

    \title{
      Funktionale und Objektorientierte Programmierung\\
      \large Dieses Skript richtet sich nach der Vorlesung von \\ Prof. Dr. rer. nat. Karsten Weihe}

    \date{\today}
    \author{Max Schmitt}
    \affil{Technische Universität Darmstadt}
    \maketitle
    %\cleardoublepage{}
    % Seiten ab jetzt mit Kopf- und Fußzeilen sowie Seitenzahl
    \pagestyle{fancy}
    % Kopfzeile
    \lhead{ }
    \chead{\leftmark}
    \rhead{}

    % Fußzeile
    \lfoot{FOP}
    \cfoot{\thepage}
    \rfoot{ %\slshape
    \date{\today} }

    % Inhaltsverzeichnis
    \begingroup
      \renewcommand*{\chapterpagestyle}{empty}
      \pagestyle{empty}
      \tableofcontents
      %\clearpage
    \endgroup

    \pagenumbering{Roman}

    \clearpage
    \pagenumbering{arabic}

    % Kapitel einbinden
    
\chapter{Grundlagen der Programmierung}
\label{c:grundlagen}
\setcounter{page}{1}
\section{Was ist Progrmamieren?}

Schauen wir zunächsteinmal, was einige der „großen Köpfe“ der
Informatik das Programmieren definieren.

\begin{quote}
	„To program is to understand“ \\
	\textit{~ Kristen Nygaard}
\end{quote}

\begin{quote}
	„Programming is a Good Medium for Expressing Poorly
	Understood and Sloppily Formulated Ideas“\\
	\textit{~ Marvin Minsky, Gerald J. Sussman}
\end{quote}

Eine Programmiersprache ist mehr als ein Hilfsmittel um einen
Computer anzuweisen, Aufgaben durchzuführen. 
Sie dient auch als \textbf{Rahmen}, innerhalb dessen wir \textbf{unsere
Ideen} über die \textbf{Problemdomäne organisieren.}

Wenn wir eine Sprache beschreiben, sollten wir
die Hilfsmittel beachten, die sie uns zum
Kombinieren von einfachen Ideen anbietet, um
komplexere Ideen zu bilden.

Jede vollwertige Programmiersprache hat drei Mechanismen,
um Prozessideen zu strukturieren:

\begin{itemize}
	\item \textbf{\textit{Primitive} Ausrücke}
		\subitem - Repräsentieren die einfachsten Einheiten der Sprache
		\subitem - Im Deutschen: jedes Wort ist ein primitiver Ausdruck
	\item \textbf{\textit{Kombinationsmittel}}
		\subitem - Zusammengesetzte Elemente werden aus einfacheren Einheiten
		konstruiert
		\subitem - Im Deutschen: Zusammensetzung mehrerer Wörter zu einem Satz.
	\item \textbf{\textit{Abstraktionsmittel}}
		\subitem - Zusammengesetzte Elemente können benannt und weiter als Einheiten manipuliert werden
		\subitem - Im Deutschen: Definition eines Begriffs („Ein Auto ist ...“), so dass der Begriff danach als „Kurzform“ für die Erklärung nutzbar ist
\end{itemize}

\textbf{Strukturierungsmechanismen in der Elektronik:}

\begin{itemize}
	\item \textbf{\textit{Primitive} Ausrücke}
		\subitem - Widerstände, Kondensatoren, Induktivitäten, Spannungsquellen, ...

	\item \textbf{\textit{Kombinationsmittel}}
		\subitem - Richtlinien für das Verdrahten der Schaltkreise
		\subitem - Standardschnittstellen (z.B. Spannungen, Strömungen) zwischen den
		Elementen. Diese Schnittstellen können auch Anforderungen an
		konkrete zulässige Werte oder Einheiten stellen („5 mA“)
		
	\item \textbf{\textit{Abstraktionsmittel}}
		\subitem - 	“Black box” Abstraktion – denke über einen Unter-Schaltkreis als eine
		Einheit: z.B. Verstärker, Regler, Empfänger, Sender, ...
\end{itemize}




    \appendix

    \end{document} %Process exited with error(s)
